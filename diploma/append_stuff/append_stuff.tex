\append{Комплектующие}
\label{appendix:stuff} % TODO починить нумерацию приложений в ref

\newcommand{\stuffitem}[3]{
    \textbf{#1}\newline
    \begin{tabular}[b]{cc}
        \includegraphics[width=5.5cm]{#2} & \begin{tabular}[b]{rl} #3 \end{tabular}
    \end{tabular}
}

\stuffitem{Бесколлекторный двигатель BGM5208-200-12}
{append_stuff/bgm5208_200_12.jpg}
{
    Количество полюсов: & 12 \\
    Внутренний диаметр: & 12мм \\
    Вес: & 85г \\
    Размер: & 63х24мм \\
    Внутр. сопротивление: & 15 Ом
}

\vspace{0.5cm}
\stuffitem{Микрокомпьютер Raspberry Pi 3B}
{append_stuff/rpi.jpg}
{
    Процессор: & Broadcom BCM2837 \\
    Количество ядер: & 4 \\
    Частота процессора: & 1.2ГГц \\
    Разрядность: & 64 бита \\
    Оперативная память: & 1 ГБ \\
    Интерфейсы: & 4xUSB, Ethernet \\
    Напряжение питания: & 5В 
}

\vspace{0.5cm}
\stuffitem{Микроконтроллер STM32 Nucleo F429ZI}
{append_stuff/stm32.jpg}
{
    Ядро: & Cortex M4 \\
    Рабочая частота: & 84 МГц \\
    Разрядность: & 32 бита \\
    Цифровых пинов: & 81 шт. \\
    Кол-во шин: & 3xi2C, 4xSPI \\
    Напряжение питания: & 5В
}

\begin{center}
    \vspace{1cm}
    Продолжение приложения А на следующей странице... \newpage
    Продолжение приложения А...
\end{center}

\vspace{0.5cm}
\stuffitem{Блок питания KMPS05}
{append_stuff/kmps05.png}
{
    Ядро: & Cortex M4 \\
    Рабочая частота: & 84 МГц \\
    Разрядность: & 32 бита \\
    Цифровых пинов: & 81 шт. \\
    Кол-во шин: & 3xi2C, 4xSPI, 3xUART \\
    Напряжение питания: & 5В
}

\vspace{0.5cm}
\stuffitem{Драйвер бесколлекторных двигателей KMBD01}
{append_stuff/kmps05.png}
{
    Ядро: & Cortex M4 \\
    Рабочая частота: & 84 МГц \\
    Разрядность: & 32 бита \\
    Цифровых пинов: & 81 шт. \\
    Кол-во шин: & 3xi2C, 4xSPI, 3xUART \\
    Напряжение питания: & 5В
}

