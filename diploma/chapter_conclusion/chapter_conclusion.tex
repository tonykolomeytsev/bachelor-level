\likechapter{\MakeUppercase{Заключение}}

В рамках работы разработана кинематическая схема четырехногого шагающего робота и рассчитан худший по нагрузке на конечности статический случай. С опорой на расчеты подобраны электроприводы удовлетворяющие крутящему моменту. Подобраны микрокомпьютер и микроконтроллер для управления роботом и вспомогательные электронные компоненты, обеспечивающие работоспособность робота. 

Спроектирован и собран прототип робота. Исследовано движение конечностей по простой траектории. Разработано управляющее программное обеспечение: на языке программирования Python написан модуль кинематических расчетов и библиотека для управления микроконтроллером; на языке С++ написана библиотека для обратной связи с микрокомпьютером. Был разработан протокол передачи данных между логическими устройствами в составе робота. Настроено окружение для тестирования и сборки кода. Выполнены все поставленные задачи. После чего был проведен анализ проделанной работы и обозначены пути модернизации прототипа.

В работе описан процесс создания прототипа шагающего робота, потенциал которого по улучшению и модернизации достаточно велик.

Доступные способы улучшения механических характеристик робота:
\begin{itemize}
    \item[1.] В проекте были использованы достаточно дешевые модели сервоприводов, которые сильно усложняют разработку управляющего программного обеспечения и снижают точность позиционирования конечностей. Замена таких приводов на более совершенные позволит производить управление не только по положению, но и по скоростям. Это позволит улучшить общие механические характеристики конечностей.
    \item[2.] Замена приводов на более совершенные позволит еще сильнее уменьшить габариты робота. 
\end{itemize}

Способы улучшения программного обеспечения:
\begin{itemize}
    \item[1.] Использовать алгоритмы машинного обучения вместо ручного задания траекторий движения конечностей.
    \item[2.] Применение искусственных нейронных сетей для обучения ходьбе в симуляции, как это делают ведущие разработчики шагающих роботов, приведенные в начале работы. 
\end{itemize}

Для улучшения электронной схемы следует начать применять печатные платы собственной разработки, включающие в свой состав микроконтроллеры, драйверы и преобразователи напряжений. Это позволит избежать проблем с коммутацией разнородных устройств друг с другом и их питанием.