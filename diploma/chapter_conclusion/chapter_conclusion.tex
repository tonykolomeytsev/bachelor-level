\likechapter{\MakeUppercase{Заключение}}

В рамках работы разработана кинематическая схема четырехногого шагающего робота и рассчитан худший по нагрузке на конечности статический случай. Подобраны электроприводы удовлетворяющие по крутящему моменту, подобраны микрокомпьютер и микроконтроллер для управления роботом, вспомогательные электронные компоненты, аккумулятор. Спроектирован и собран прототип робота. Разработано управляющее программное обеспечение. На языке программирования Python: модуль кинематических расчетов, библиотека для управления микроконтроллером. На языке С++: библиотека для обратной связи с микрокомпьютером. Был с нуля разработан протокол передачи данных между логическими устройствами в составе робота.

\section*{Критический анализ пробеланной работы}

В работе были использованы не самые лучшие модели сервоприводов. Разработка управляющего программного обеспечения сильно усложняется при их использовании. В ходе дальнейшего развития проекта планируется замена приводов на их более дорогие, но более точные по позиционированию аналоги. Также нужно отметить, что в ходе развития проекта планируется применение искусственных нейронных сетей для обучения ходьбе в симуляции, как это делают ведущие разработчики шагающих роботов, приведенные в начале работы.

В целом, в ходе работы были применены некоторые темы из численных методов, вычислительной механики. Были применены основы кинематики, тригонометрии, для решения возникших задач. Для реализации ПО были изучены многие сложные вопросы касающиеся программирования.