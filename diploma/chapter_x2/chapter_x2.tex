\chapter{\MakeUppercase{Четырехногий шагающий робот}}
\section{Конструкция шагающего робота}
Конструкция максимально простая. Корпус робота содержит в себе всю силовую и логическую электронику. К корпусу крепится четыре ноги. Каждая нога является манипулятором с тремя степенями свободы. 

После того как было решено использовать эту кинематику, был проведен анализ механики системы и спроектирована твердотельная 3D-модель робота.

\begin{center}
    [СКРИНШОТ СБОРКИ]
\end{center}

Отсутствие стоп у робота позволяет уменьшить количество степеней свободы.

\section{Строение ноги}

На рисунке X представлена конструкция ноги робота. Как можно заметить, все три электропривода смещены к основанию ноги. Сделано это из практических побуждений. Такое расположение электроприводов позволяет уменьшить моменты инерции звеньев за счет максимального их облегчения. Такая конструкция позволяет использовать редуктор с малым передаточным числом. Это в свою очередь приводит к тому, что резко возрастает надежность привода, растёт степень обратной тяги (backdrivable) и способность распознавать проприоцептивную силу \cite{Seok2012}. Все это идеальные атрибуты для создания шагающих роботов.

\begin{table}[ht]
    \caption{Список использованных комплектующих}
    \label{tab_stuff}
    \centering
    \begin{tabular}{|c|c|c|}
        \hline Наименование     & Назначение        & Кол-во \\
        \hline Raspberry Pi 3B  & Микрокомпьютер    & 1 \\
        \hline STM32 Nucleo     & Микроконтроллер   & 1 \\
        \hline BGM5208-200-12   & Бесколлекторный трехфазный двигатель & 6 \\
        \hline KMPS05           & Блок питания      & 1 \\
        \hline KMBD01           & Драйвер BLDC      & 6 \\
        \hline TB6600           & Драйвер ШД        & 1 \\
        \hline Nema 11          & Шаговый двигатель & 1 \\
        \hline
    \end{tabular}
\end{table}

\section{Электропривод}

В данном проекте используются трёхфазные синхронные бесколлекторные двигатели с внешним ротором на постоянных магнитах. В зарубежной литературе их обозначают BLDC.

BLDC моторы более универсальны, чем ДПТ, из-за более высокой скорости вращения и более высокого крутящего момента \cite{Derammelaere2016}. Они также более компактны, что делает их идеальным вариантом для встраивания в сложные, ограниченные по габаритам системы. В основном такие двигатели начали применяться сначала в жестких дисках, системах охлаждения электроники. Позже, мощные BLDC моторы стали применяться в гибридных и электромобилях. В нынешнее время, в связи с развитием микроэлектроники, эти моторы стали применяться в промышленных задачах позиционирования и приведения, в робототехнике и мультикоптерах.

\section{Редуктор}

\section{Тестирование ноги}

\section{Постановка задачи}

Цель работы - разработка двуногого шагающего робота и алгоритма его управления. Во время выполнения работы будут решаться следующие инженерные задачи:

\begin{itemize}
    \item Анализ механики системы (статика, кинематика, динамика)
    \item Подбор комплектующих
    \item Разработка драйвера для BLDC двигателей
    \item Создание трехмерной твердотельной модели системы
    \item Сборка робота, пусконаладочные работы
    \item Программирование системы управления
\end{itemize}

Подробная таблица с описанием технических характеристик всех комплектующих в приложении А.%\aref{appendix:stuff}.