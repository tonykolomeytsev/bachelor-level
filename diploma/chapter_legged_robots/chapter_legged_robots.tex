\chapter{\MakeUppercase{Постановка задачи и план решения}}

\section{Задачи работы}

В рамках работы рассматривается разработка шагающего четырехногого робота, с упором на проблему проектирования конечностей и разработки программного обеспечения (ПО) для управления движением.

Для выполнения работы требуется:
\begin{itemize}
    \item Разработать общую кинематическую схему робота.
    \item Рассчитать худший (по нагрузке) статический случай для конечностей четырехногого робота.
    \item Подобрать электроприводы удовлетворяющие по крутящему моменту.
    \item Подобрать остальные комплектующие (управляющая электроника, детали корпуса, источник питания, проводка), входящие в состав робота и обеспечивающие его автономную работу.
    \item Спроектировать и собрать прототип робота.
    \item Разработать ПО для управления робота.
\end{itemize}

\section{Краткое описание кинематической схемы}

После изучения существующих на рынке роботов, было решено реализовать в прототипе наиболее распространенную конструкцию. Можно заметить что у всех приведенных моделей роботов почти одинаковая кинематика, характеризующаяся четырьмя конечностями, у каждой из которых по три степени свободы. Поэтому с опорой на опыт разработчиков роботов, о которых написано во введении, в качестве кинематической схемы прототипа была выбрана схема, продемонстрированная на рисунке \ref{fig:kin_scheme1}. 

Кинематика конечности подробнее рассматривается далее. Также далее конечности допустимо называть ногами, а точку $ A $ -- стопой робота. Изображенный на рисунке \ref{fig:kin_scheme1} параллелепипед -- это корпус, тело робота.

\begin{figure}[ht]
    \centering
    \includegraphics[scale=1.16]{chapter_legged_robots/figure1.png}
    \caption{Кинематика четырехногого робота. Нумерация ног произвольная}
    \label{fig:kin_scheme1}
\end{figure}

% \fixme ИЛЛЮСТРАЦИЯ

\subsection{Определение максимальной нагрузки на приводы конечностей}

Расчет максимальной нагрузки производится при известных размерах комплектующих робота. Это объясняется тем, что чертежи изделия (рисунок \ref{fig:final_render}) были подготовлены до выбора приводов.

\begin{figure}[h]
    \centering
    \includegraphics[width=0.85\textwidth]{chapter_mechanics_construction/figure20.png}
    \caption{Сборочный чертеж робота.}
    \label{fig:final_render}
\end{figure}

У большей части доступных на рынке сервоприводов одинаковые габариты, форма корпуса и даже расположение крепежных отверстий. Разница лишь в крутящих моментах, характеризующихся передаточным числом редуктора. Отсюда следует, что чертежи таких приводов ничем не отличаются друг от друга и это позволяет размещать их в сборочном чертеже робота до того как станет известна нагрузка на приводы. После того как становится известен точный набор комплектующих и их вес, а также длины конечностей, мы можем выбрать сервопривод, удовлетворяющий требуемому крутящему моменту.

Найдем максимальную статическую нагрузку на приводы конечностей, рассмотрев наихудшую статическую конфигурацию робота. «Худшей» конфигурацией называется такая, при которой одному или нескольким приводам нужно приложить максимальный момент для поворота звена конечности в нужную сторону.

Конструкция ноги робота не позволяет разогнуть ее полностью. Поэтому можно привести длины звеньев конечности спроецированные на горизонтальную ось для того чтобы провести приблизительные расчеты:
\begin{align*}
    l_1 &\approx 131\: мм, \\
    l_2 &\approx 134.5\: мм
\end{align*}

Худшим случаем является случай, при котором робот лежит на «животе» с выпрямленными конечностями. Чтобы подогнуть под себя конечность, нужно будет преодолеть момент $M_{худш}$ с учетом массы тела робота $m_T$. При этом максимальный крутящий момент потребуется приводу, находящемуся в первом узле, или можно сказать, управляющему первой степенью свободы.

\begin{figure}[ht]
    \centering
    \includegraphics[scale=1]{kin2.png}
    \caption{Иллюстрация худшего статического случая}
\end{figure}
% на рисунке показать реакцию опоры и mg

При расчёте в первом приближении можно пренебречь трением и весами звеньев. Также можно учесть, что нагрузка, создаваемая массой тела робота будет распределена равномерно по всем четырем ногам. Это значит что на одну ногу будет приходится четверть массы тела робота.

Самой тяжелой частью конструкции оказались каркас корпуса, в составе которого много алюминиевых профилей, и аккумулятор. Самые легкие составляющие конструкции -- это электронные компоненты. Суммарно все комплектующие корпуса весят $3.2$ кг. 

При таком расчете можно считать, что каждой ноге надо будет «поднять» около $ 0.8 $ кг веса. Тогда в худшем случае, приложенный момент вычисляется просто:
\begin{equation} \label{eq:statics1}
    M_{худш}= \frac 1 4 (l_{1}+l_{2}) m_T g, 
\end{equation}
\noindent здесь $l_1$ и $l_2$ - длины звеньев ноги.

При расчете по формуле \ref{eq:statics1} требуемый крутящий момент равен примерно $ 2.1  \: Н \cdot м$. После проведенного поиска среди доступных на рынке сервоприводов, был выбран сервопривод с крутящим моментом $ 2.5 \: Н \cdot м $. С таким сервоприводом максимально допустимый вес робота будет примерно равен $ 3.8 \: кг $.
