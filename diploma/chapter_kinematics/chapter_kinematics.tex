\chapter{\MakeUppercase{Кинематика конечностей робота}}
\section{Общее положение}
Изначально планировалось, что конечность робота в общем виде будет представлять манипулятор с тремя степенями свободы. Такая конструкция множество раз рассмотрена другими людьми, существует аналитическое решение прямой и обратной задачи.

Проблема такой кинематики в механической сложности ее реализации с точки зрения конструктора. Из-за особенностей и требований, описанный выше \fixme конструкция ноги получилась, как на рисунке ниже.

Для упрощения понимания ниже приведена трехмерная кинематическая схема конечности.

В составе конечности присутствует непрямой кинематический четырехзвенник (три из четырех сторон имеют разные длины). Это усложнило кинематическую задачу, сделало ее менее тривиальной. Как оказалось далее, наличие четырехзвенной передачи сделало невозможным нахождение аналитического решения обратной задачи кинематики.

На кинематической схеме отмечены величины, численные значения которых приведены ниже:

\begin{align*}
    a_0&=0.01 м \\
    a_1&=46.22 \times 10^{-3} м \\
    a_2&=20 \times 10^{-3} м \\
    a_3&=44 \times 10^{-3} м \\
    a_4&=20 \times 10^{-3} м \\
    a_5&=27 \times 10^{-3} м \\
    a_6&=87 \times 10^{-3} м \\
    a_7&=107 \times 10^{-3} м \\
    a_8&=12.62 \times 10^{-3} м \\
    a_9&=24.5 \times 10^{-3} м \\
    a_10&=110 \times 10^{-3} м
\end{align*}

Диапазоны рабочих углов следующие:
\begin{align*}
    \varphi_1&=0 \dots \frac \pi 8 \\
    \varphi_2&=-\frac \pi 2 \dots 0 \\
    \varphi_3&=-\frac \pi 4 \dots \frac \pi 4
\end{align*}


\section{Общее решение четырёхзвенника}


\section{Прямая кинематика}


\section{Обратная кинематика}


\section{Оптимизация численного решения обратной задачи}