\chapter{\MakeUppercase{Кинематика конечностей робота}}
\section{Общее положение} \label{sec:kin_general}
Изначально планировалось, что конечность робота в общем виде будет представлять манипулятор с тремя степенями свободы. Такая конструкция множество раз рассмотрена другими людьми, существует аналитическое решение прямой и обратной задачи.

Проблема такой кинематики в механической сложности ее реализации с точки зрения конструктора. Из-за особенностей и требований, описанных в пункте \ref{sec:leg_design}, конструкция ноги получилась, как на рисунке ниже.
\begin{figure}[h]
    \centering
    \includegraphics[scale=0.5]{chapter_kinematics/figure2.png}
    \caption{Чертеж конструкции ноги}
    \label{}
\end{figure}

Для упрощения понимания ниже приведена трехмерная кинематическая схема конечности.
\begin{figure}[h]
    \centering
    \includegraphics[scale=0.5]{chapter_kinematics/figure1.png}
    \caption{Трехмерная кинематическая схема ноги робота}
    \label{fig:kin_scheme}
\end{figure}

В составе конечности присутствует непрямой кинематический четырехзвенник (три из четырех сторон имеют разные длины). Это усложнило кинематическую задачу, сделало ее менее тривиальной. Как оказалось далее, наличие четырехзвенной передачи сделало невозможным нахождение аналитического решения обратной задачи кинематики. 

На кинематической схеме отмечены величины, численные значения которых приведены ниже:
\begin{align*}
    a_0&=0.01 м \\
    a_1&=46.22 \times 10^{-3} м \\
    a_2&=20 \times 10^{-3} м \\
    a_3&=44 \times 10^{-3} м \\
    a_4&=20 \times 10^{-3} м \\
    a_5&=27 \times 10^{-3} м \\
    a_6&=87 \times 10^{-3} м \\
    a_7&=107 \times 10^{-3} м \\
    a_8&=12.62 \times 10^{-3} м \\
    a_9&=24.5 \times 10^{-3} м \\
    a_10&=110 \times 10^{-3} м
\end{align*}

Заметим, что расстояния $ a_5 $ и $ a_9 $ различаются! Диапазоны рабочих углов следующие:
\begin{align*}
    \varphi_1&=0 \dots \frac \pi 8 \\
    \varphi_2&=-\frac \pi 2 \dots 0 \\
    \varphi_3&=-\frac \pi 4 \dots \frac \pi 4
\end{align*}

Если зафиксировать первую степень свободы, $ \varphi_1 = 0 $, тогда рабочая область ноги будет лежать в плоскости $ XZ $. Выглядеть рабочая область с зафиксированным $ \varphi_1 $ будет следующим образом:
\begin{figure}[h]
    \centering
    \includegraphics[scale=0.5]{chapter_kinematics/figure3.png}
    \caption{Сечение рабочей области конечности робота}
    \label{}
\end{figure}

Для построения рабочей области использованы уравнения прямой кинематики, которые будут выведены в пункте \ref{sec:direct_kinematics}.

\section{Общее решение четырёхзвенника} \label{sec:pre_direct_kin}

При построении аналитического решения четырехзвенной передачи было важно подобрать функции так, чтобы для рабочих диапазонов углов $ \varphi_1, \varphi_2, \varphi_3 $ не возникало переходов через ноль, или через бесконечно большие числа. Важно избежать использования кусочных функций и условных операторов для упрощения программирования.
Малые диапазоны углов в рассматриваемой конечности упрощают эту задачу.

На рисунке \ref{fig:scheme_four} приведена схема четырехзвенника. Найдем зависимость $ \varphi_4 $ от $ \varphi_3 $. Здесь $ l= a_4 + a_6=a_7 $ введен для упрощения расчётных формул, т.к. две стороны четырехзвенника имеют одну длину.
\begin{figure}[h]
    \centering
    \includegraphics[scale=1]{chapter_kinematics/figure4.png}
    \caption{Четырехзвенник}
    \label{fig:scheme_four}
\end{figure}

Сначала найдем диагональ $d$ по теореме косинусов, для того чтобы выразить противолежащие углы:
\begin{align}
    d = \sqrt{a_5^2 + l^2 - 2a_5 l \cos(\varphi_3)} \label{eq:kin_1}
\end{align}

\noindent С помощью уравнения \ref{eq:kin_1} выразим углы:
\begin{align}
    \gamma &= \text{arcsin}\left(\frac{a_5}{d}\cos \varphi_3\right) \label{eq:gamma} \\
    \delta &= \text{arccos}\left(\frac{d^2+a_9^2-l^2}{2 a_9 l}\right) \label{eq:delta}
\end{align}

\noindent И искомый угол $ \varphi_4 $ будет находиться следующим образом:
\begin{align}
    \varphi_4 = \pi - \gamma - \delta \label{eq:phi4_simple}
\end{align}

\noindent Подставим результаты \ref{eq:kin_1}, \ref{eq:gamma}, \ref{eq:delta} в уравнение \ref{eq:phi4_simple} таким образом получим  зависимость $ \varphi_4 $ от $ \varphi_3 $:
\begin{multline}
    \varphi_4 = \pi - \text{arcsin}\left(\frac{a_5}{\sqrt{a_5^2 + l^2 - 2a_5 l \cos(\varphi_3)}}\cos \varphi_3\right) - \\ - \text{arccos}\left(\frac{a_5^2 + l^2 - 2a_5 l \cos(\varphi_3)+a_9^2-l^2}{2 a_9 l}\right) 
\end{multline}

\section{Прямая кинематика}\label{sec:direct_kinematics}
Используя найденную в пункте \ref{sec:pre_direct_kin} связь построим решение прямой задачи кинематики по координатам точки $ A $. Трехмерная кинематическая схема приведена на рисунке \ref{fig:kin_scheme}. Зафиксируем сначала первую степень свободы ($ \varphi_1 = 0 $) и найдем координаты точки $ A $ в плоскости, параллельной плоскости $ XZ $:
\begin{multline}
    X_A=a_2+a_3\cos(\varphi_2)+a_6\sin(\varphi_2)-\\-a_8 \cos(\varphi _2+\varphi _4)+(a_9+a_{10}) \sin(\varphi _2+\varphi _4) 
\end{multline}
\begin{multline}
    Z_A=a_1-a_3 \sin(\varphi _2)+a_6 \cos(\varphi _2)+\\+a_8 \sin(\varphi _2+\varphi _4)+(a_9+a_{10}) \cos(\varphi _2+\varphi _4)
\end{multline}

\noindent Если мы предположим что $ \varphi_1 = \frac \pi 2 $, тогда конечность будет <<поднята над землей>> в таком положении. Координата $ Z $ точки $ A $ станет равна нулю, координата $ Y $ увеличится. Так как от $ \varphi_1 $ зависят только координаты $Y$ и $Z$, получим следующие уравнения для координат точки $A$:
\begin{multline}
    X_A=a_2+a_3\cos(\varphi_2)+a_6\sin(\varphi_2)-\\-a_8 \cos(\varphi _2+\varphi _4)+(a_9+a_{10}) \sin(\varphi _2+\varphi _4) 
\end{multline}
\begin{multline}
    Y_A=a_0+\sin(\varphi_1)[a_1-a_3 \sin(\varphi _2)+a_6 \cos(\varphi _2)+\\+a_8 \sin(\varphi _2+\varphi _4)+(a_9+a_{10}) \cos(\varphi _2+\varphi _4)]
\end{multline}
\begin{multline}
    Z_A=\cos(\varphi_1)[a_1-a_3 \sin(\varphi _2)+a_6 \cos(\varphi _2)+\\+a_8 \sin(\varphi _2+\varphi _4)+(a_9+a_{10}) \cos(\varphi _2+\varphi _4)]
\end{multline}

\noindent Таким образом прямая задача кинематики решена.

Следует отметить, что из-за того что $ a_9 < a_5 $, диапазон углов $ \varphi_4 $ получился шире, чем у $ \varphi_3 $. Это наглядно можно продемонстрировать на проекции математической модели ноги на плоскость $ XZ $. Здесь синей точкой отмечено положение реальное положение точки $ A $. Красной точкой отмечено положение точки $ A $ в том случае, если бы в четырехзвеннике $ a_5 $ был равен $ a_9 $. В крайних положениях третье звено конечности поворачивается на угол немного больший, чем поворачивается вал сервопривода.
\begin{figure}[ht]
    \centering
    % левая картинка
    \begin{subfigure}[b]{0.45\textwidth}    
        \centering
        \includegraphics[scale=0.4]{chapter_kinematics/figure5.png}
        \caption{}
    \end{subfigure}
    % правая картинка  
    \begin{subfigure}[b]{0.45\textwidth}
        \centering
        \includegraphics[scale=0.4]{chapter_kinematics/figure6.png}
        \caption{}
    \end{subfigure}
     
    \caption{(a) Третье звено полностью <<разогнуто>>; (б) Третье звено полностью <<согнуто>>.}
    \label{fig:leg_model}
\end{figure}

Таким образом за счет использования четырехзвенной передачи, увеличена рабочая область конечности. Ниже на графике зеленая область -- рабочая область при $ a_5 = a_9 $, синяя область -- расширение рабочей области при $ a_5 > a_9 $.
\begin{figure}[h]
    \centering
    \includegraphics[scale=0.8]{chapter_kinematics/figure7.png}
    \caption{Сравнение рабочих областей}
    \label{}
\end{figure}

\section{Обратная кинематика}
Как уже упоминалось ранее в пункте \ref{sec:kin_general}, аналитического решения для задачи обратной кинематики нет. Это значит что нужно использовать численные методы решения.

Будем искать решение с помощью метода Ньютона для систем нелинейных уравнений. В общем виде который будет выглядеть следующим образом:

Пусть дана система из $ n $ нелинейных уравнений с $ n $ неизвестными.
\[
\left\{ 
\begin{array}{c}
    f_1(x_1, \dots, x_n) = 0 \\
    f_2(x_1, \dots, x_n) = 0 \\
    \vdots \\
    f_1(x_1, \dots, x_n) = 0 \\
\end{array} 
\right.
\]
\noindent Где $ f_i(x_1,\dots,x_n): \mathbb{R}^n \rightarrow \mathbb{R}, i=1,\dots, n $ -- нелинейные функции, определенные и непрерывно дифференцируемые в некоторой области $ G \subset \mathbb{R}^n $. Для записи в векторном виде введем величины:
\begin{align*}
    \overline{x} &= [x_1, x_2, \dots x_n]^T \\
    F(x) &= [f_1(x), f_2(x),\dots,f_n(x)]^T = 0
\end{align*}

Нужно найти такой вектор $ \overline{x}^*=[x_1^*, x_2^*, \dots x_n^*]^T $, чтобы было верно равенство $ F(\overline{x}^*) = 0 $. Формула для нахождения решения итеративным методом выглядит следующим образом:
\begin{align*}
    x^{(k+1)}=x^{(k)}-J^{-1}(x^{(k)}) \times F(x^{(k)})
\end{align*}

\noindent Где $ k=1,2\dots $, а $ J $ -- матрица Якоби:
\begin{align*}
    J = \begin{bmatrix}
        \frac{\partial f_1(x_1)}{\partial x_1} & \dots & \frac{\partial f_1(x_n)}{\partial x_n} \\
        \vdots & \ddots & \vdots \\
        \frac{\partial f_n(x_1)}{\partial x_1} & \dots & \frac{\partial f_n(x_n)}{\partial x_n}
    \end{bmatrix}
\end{align*}

\section{Оптимизация численного решения обратной задачи}