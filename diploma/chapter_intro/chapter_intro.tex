\likechapter{Введение}

Шагающие роботы - это класс роботов, имитирующих движения людей и животных. Миллионы лет эволюциии показывают, что передвижение при помощи ног это наиболее эффективный способ быстро приспосабливаться к плохим, неровным поверхностям. Люди пытались описывать ходьбу шагающими механизмами, математическими формулами. На сегодняшний день понятно, что ни один из этих двух способов к результатам, применимым на практике, не приведет. От шагающей системы требуется приспособиться к тем условиям, в которых она раньше не была.

\textbf{Актуальность работы}

Задача создания шагающих роботов общего назначения с повышенной степенью автономности и надежности сегодня актуальна, как никогда ранее. Всё в большей степени людей стараются заменять роботами для работ вроде общего тех. осмотра помещений, исследования местности вдали от дорог и цивилизации, помощи в устранении последствий катастроф. Открытость методик, исходных кодов и готовых рассчитанных моделей приведет к массовой разработке шагающих роботов не только крупными предприятиями, но и мелкими разработчиками.

\textbf{Рынок шагающих роботов}

Классифицировать шагающие машины можно не только по количеству ног. Некоторые роботы комплектуются также и колесами, для увеличения скорости передвижения по ровным поверхностям. 

На текущий момент можно найти огромное количество шагающих роботов. Гексаподы, робо-пауки и т.д. Но среди сотен моделей можно выделить всего несколько роботов, ходьба которых максимально приближена к животной.

Из четырёхногих роботов лучшие результаты показывают:
\begin{itemize}
    \item Роботы Spot и Spot-Mini от компании Boston Dynamics
    \item Робот Mini-Cheetah от студентов MIT
    \item Робот ANYmal от студентов Цюрихского университета
    \item Робот HyQReal от IIT (Итальянского технологического института)
    \item Роботы LaikaGo и AlienGo от компании Unitree Robotics
\end{itemize}

Среди двуногих можно выделить такие проекты, как:

\begin{itemize}
    \item Робот Digit от компаний Agility Robotics и Ford
    \item Робот Cassie от компании Agility Robotics
    \item Робот Atlas от компании Boston Dynamics
\end{itemize}

Всех выше перечисленных роботов объединяет одно важное свойство - они обучены ходить. Алгоритм их ходьбы не описан статическими константами в коде, он создан при помощи методов машинного обучения. Благодаря этому все они показывают высокую степень мобильности и адаптивности к окружающей среде. Их перемещения похожи на то, как могли бы перемещаться животные со схожим механическим строением тела.

Будущее шагающих роботов именно за машинным обучением, а если быть точнее, за глубоким обучением c подкреплением (reinforcement learning). Модели, обученные в симуляции показывают более высокий КПД при перемещении и меньшее потребление тока, чем модели ходьбы, описанные человеком вручную \cite{Hwangbo2019}. 