\chapter{\MakeUppercase{Анализ проделанной работы}}

В работе описан процесс создания прототипа шагающего робота, потенциал которого по улучшению и модернизации достаточно велик.

Доступные способы улучшения механических характеристик робота:
\begin{itemize}
    \item[1.] В проекте были использованы достаточно дешевые модели сервоприводов, которые сильно усложняют разработку управляющего программного обеспечения и снижают точность позиционирования конечностей. Замена таких приводов на более совершенные позволит производить управление не только по положению, но и по скоростям. Это позволит улучшить общие механические характеристики конечностей.
    \item[2.] Замена приводов на более совершенные позволит еще сильнее уменьшить габариты робота. 
\end{itemize}

Способы улучшения программного обеспечения:
\begin{itemize}
    \item[1.] Вместо ручного задания траекторий движения конечностей можно использовать алгоритмы машинного обучения.
    \item[2.] Применение искусственных нейронных сетей для обучения ходьбе в симуляции, как это делают ведущие разработчики шагающих роботов, приведенные в начале работы. 
\end{itemize}

Для улучшения электронной схемы следует начать применять печатные платы собственной разработки, включающие в свой состав микроконтроллеры, драйверы и преобразователи напряжений. Это позволит избежать проблем с коммутацией разнородных устройств друг с другом и их питанием.